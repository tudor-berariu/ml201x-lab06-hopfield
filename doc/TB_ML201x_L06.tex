\documentclass[12pt]{article}

\usepackage[utf8x]{inputenc}%
\usepackage[romanian]{babel}%
\usepackage{color}%
\usepackage{algorithm}%
\usepackage{algorithmic}%
\usepackage{hyperref}%
\usepackage{amsmath}%
\usepackage{graphicx}%
\usepackage{ftnxtra}%

\usepackage{algorithm}
\usepackage{algorithmic}
\usepackage{float}

\floatname{algorithm}{Algoritmul}
\renewcommand{\algorithmicforall}{\textbf{pentru toate}}
\renewcommand{\algorithmicdo}{\textbf{execută}}
\renewcommand{\algorithmicuntil}{\textbf{până când}}
\renewcommand{\algorithmicend}{\textbf{termină}}
\renewcommand{\algorithmicif}{\textbf{dacă}}
\renewcommand{\algorithmicelse}{\textbf{altfel}}
\renewcommand{\algorithmicfor}{\textbf{ciclu}}
\renewcommand{\algorithmicthen}{\textbf{atunci}}
\renewcommand{\algorithmicrepeat}{\textbf{repetă}}
\renewcommand{\algorithmicendfor}{\algorithmicend\ \algorithmicfor}
\renewcommand{\algorithmicrequire}{\textbf{Intrări:}}
\renewcommand{\algorithmicensure}{\textbf{Ieșire:}}


\title{Învățare Automată - Laboratorul 6 \\ \textbf{Rețele Hopfield}}%
\author{Tudor Berariu \\ {\small Laboratorul AIMAS} \\ {\small
    Facultatea de Automatică și Calculatoare}}%

\begin{document}
\maketitle%

\section{Scopul laboratorului}
\label{sec:goals}

Scopul laboratorului îl reprezintă înțelegerea și implementarea unei
rețele Hopfield.

\section{Rețele Hopfield}
\label{sec:intro}

O rețea Hopfield este o rețea \emph{asincronă} cu $n$ neuroni, total
conectată (fiecare neuron are intrările conectate la ieșirile tuturor
celorlalți $n-1$ neuroni). O rețea este asincronă facă fiecare unitate
(neuron) își actualizează starea la momente de timp aleatoare,
independent de timpii de actualizare ale cerlolalte unități.

Într-o rețea Hopfield \emph{funcția de activare} (actualizare) pentru
un neuron este cea din Formula~\ref{eq:activation}.

\begin{equation}
  \label{eq:activation}
  x_i \longleftarrow sgn \Big( \displaystyle\sum_{j=1}^{n}w_{ij}x_j \Big)
\end{equation}

Utilizând principiul lui Hebb, o rețea Hopfield poate fi folosită ca o
memorie asociativă pentru a reține un număr de șabloane binare (un
șablon va fi format din $n$ valori din mulțimea $\lbrace -1, 1
\rbrace$). Ponderile unei rețele Hopfield se calculează pe baza celor
$m$ șablaone $(\mathbf{s}^i,1 \le i \le m)$ conform
Formulei~\ref{eq:hebb} (învățare hebbiană).

\begin{equation}
  \label{eq:hebb}
  \mathbf{W} = \displaystyle\sum_{i=1}^m \mathbf{s}^i \cdot (\mathbf{s}^i)^T
  \; - \; m\mathbf{I}
\end{equation}

Atenție: $w_{ii} = 0, \forall i \in \lbrace 1 \ldots n \rbrace$.

Pentru a folosi rețeaua ca un clasificator, după ce aceasta a fost
\emph{antrenată} (Formula~\ref{eq:hebb}), se utilizează
Algoritmul~\ref{alg:recognition}.

\begin{algorithm}
  \caption{Recunoașterea șabloanelor}
  \label{alg:recognition}
  \begin{algorithmic}[1]
    \REQUIRE ponderile rețelei $W$, șablonul nou $t$%
    \ENSURE șablonul recunoscut $s$%
    \STATE $s \longleftarrow t$%
    \REPEAT%
    \STATE alege aleator un neuron $i$%
    \STATE $s_i \longleftarrow sgn\Big(\displaystyle\sum_{j=1}^{n} w_{ij}s_j\Big)$%
    \UNTIL{stările de activare ale neuronilor nu se mai schimbă}%
  \end{algorithmic}
\end{algorithm}

\section{Cerințe}
\label{sec:tasks}

În cadrul acestui laborator se va implementa o rețea Hopfield pentru a
recunoate imagini ce reprezintă cifrele de la 0 la 9. O imagine este
formată din $10\times12$ pixeli, precum cea de mai jos
\begin{verbatim}
__xxxxxxxx__
__xxxxxxxxx_
________xxx_
________xxx_
___xxxxxxx__
___xxxxxxx__
________xxx_
________xxx_
__xxxxxxxxx_
__xxxxxxxx__
\end{verbatim}

Limbajul de programare este la alegere, dar pentru Python, C++ și
Octave există un schelet de cod cu următoarele funcții implementate:

\begin{itemize}
\item \texttt{read\_digits(m)} - citește primele $m$ șabloane din
  fișierul \texttt{digits}
  \begin{itemize}
  \item fiecare șablon este memorat ca un vector / listă de lungime
    120 conținând doar valori 1 și -1
  \item rezultatul funcției este o colecție de $m$ astfel de vectori
    (o matrice $m \times 120$ cu șabloanele pe linii)
  \end{itemize}
\item \texttt{print\_digit(d)} - afișează un șablon
\item \texttt{add\_noise(pattern, noise)} - întoarce un șablon nou
  adăugând zgomot unui șablon original
  \begin{itemize}
  \item $0 \le noise \le 1$ reprezintă probabilitatea cu care un
    \emph{pixel} al șablonului este schimbat din 1 în -1 și invers
  \end{itemize}
\end{itemize}

\subsection*{Cerințe}
\label{sec:tasks10}

Implementați următoarele funcții:
\begin{enumerate}
\item \texttt{compute\_weights(patterns)} - în care calculați
  ponderile unei rețele Hopfield pentru șabloanele primite.
\item \texttt{converge(weigths, new\_pattern)} - în care implementați
  algoritmul de recunoaștere a șabloanelor pentru rețele Hopfield.
\item \texttt{compute\_accuracy(patterns, noise)} - în care construiți
  o rețea Hopfield care să memoreze șabloanele \texttt{patterns},
  generați imagini noi adăugând zgomot celor originale și testați
  capacitatea rețelei de a reface șablonul inițial.
\end{enumerate}

\subsection*{Bonus}
\label{sec:bonus}

\begin{enumerate}
\item Construiți un grafic al variației acurateței clasificării în
  funcție de nivelul de zgomot. Fixați mai multe valori pentru numărul
  de șabloane învățate și adăugați câte o linie în grafic pentru
  fiecare.
\item Construiți un grafic al variației acurateței clasificării în
  funcție de numărul de șabloane învățate. Fixați mai multe valori
  pentru nivelul de zgomot adăugat la recunoaștere și adăugați câte o
  linie în grafic pentru fiecare dintre acestea.
\end{enumerate}

\subsection*{Arhiva}
\label{zip-archive}

Arhiva conține schelet de cod în Octave (sau Matlab), C++11 și Python.

\end{document}
